% Intended LaTeX compiler: pdflatex
\documentclass[conference,letter,10pt,final]{IEEEtran}
\usepackage[utf8]{inputenc}
\usepackage[T1]{fontenc}
\usepackage{graphicx}
\usepackage{grffile}
\usepackage{longtable}
\usepackage{wrapfig}
\usepackage{rotating}
\usepackage[normalem]{ulem}
\usepackage{amsmath}
\usepackage{textcomp}
\usepackage{amssymb}
\usepackage{capt-of}
\usepackage{hyperref}
\usepackage[utf8]{inputenc}
\usepackage[T1]{fontenc}
\date{}
\title{Social Navigation Evaluation Scenarios and Why They're Kinda Bogus}
\hypersetup{
 pdfauthor={Alex Day},
 pdftitle={Social Navigation Evaluation Scenarios and Why They're Kinda Bogus},
 pdfkeywords={},
 pdfsubject={},
 pdfcreator={Emacs 27.2 (Org mode 9.5)}, 
 pdflang={Brazilian}}
\begin{document}


\makeatletter
\let\orgtitle\@title
\makeatother

\title{\orgtitle}

\author{
\IEEEauthorblockN{Alex Day}

\IEEEauthorblockA{School of Computing, Clemson University}

}

\maketitle

\begin{abstract}

Put the abstract here.

\end{abstract}

\section{Introduction}
\label{sec:org8b7aafd}

This paper is organized as follows: section \ref{sec.background} presents the background and current evaluation techniques.

\section{Background}
\label{sec.background}
Social navigation algorithms aim to solve a path planning problem in a scenario
both humans and robots. The introduction of human agents radically changes the
objectives of, and, consequentially, the methods for evaluating, these algorithms.
The main goal shifts from maximizing the global efficiency of all agents to
avoiding excessive interaction with and hindrance of the human agents.

\begin{itemize}
\item SNA solve a modified path planning problem
\begin{itemize}
\item Heterogeneous simulation (human and robits)
\item This changes the game completely for evaluation
\begin{itemize}
\item Previous Goal: Maximize global efficiency
\item New Goal: Don't mess with the humans but still get to your goal fast
\end{itemize}
\item Current approaches
\begin{itemize}
\item DRL
\begin{itemize}
\item \cite{chen2017socially}
\end{itemize}
\item DL
\begin{itemize}
\item NaviGAN
\end{itemize}
\end{itemize}
\end{itemize}
\end{itemize}

Methods of evaluation for social navigation algorithms can be broadly separated
into three distinct classes. The first is to utilize existing pedestrian
interaction datasets. The agents initial positions are set to the pedestrian
positions at some time \(\tau\). As the both the agents and pedestrians positions
progress as time goes on the difference between the two can be measured. This
can be thought of as a measure of how human-like the agents movements are. This
is attractive method because there is a wealth of pedestrian data online.
However, the pedestrians in these datasets are not in the mindset of interacting
with robots, therefore the results from this measure might not map directly to
the goodness of an algorithm.

\begin{itemize}
\item Simulation Techniques
\begin{itemize}
\item Utilizing existing pedestrian-pedestrian interaction
\begin{itemize}
\item Datasets are widely available
\item Deterministic metrics for evaluation (same trajectories each time)
\item Pedestrians are trying to interact with pedestrians, vibes are all wrong
\item \cite{yao19following}
\end{itemize}
\end{itemize}
\item Completely simulating pedestrians
\begin{itemize}
\item Can run trials until your CPU explodes
\item Dependent on how good your pedestrian algorithms are and how realistic the simulation is
\end{itemize}
\item Using real humans to interact, even in simulation or real life
\begin{itemize}
\item Ideal
\end{itemize}

\item Datasets are required, no matter the technique choice
\begin{itemize}
\item Crowd datasets are required for 1. and there is a finite amount
\item For the last two
\begin{itemize}
\item Realistic start and goal positions are required
\item Hypothesis: These start and goal positions are normally bogus
\begin{itemize}
\item Circle is bogus
\item Two groups not much better
\end{itemize}
\end{itemize}
\end{itemize}
\end{itemize}

\subsection{Human in the Loop}
\label{sec:orgfff1880}
\cite{tsoi20seanep}

\subsection{Simulation}
\label{sec:org3f2030d}
Evaluating the navigation algorithm in the presence of simulated  \cite{tsoi20sean}
**

\section{Conclusion}
\label{sec:org3669fc5}
\bibliographystyle{IEEEtran}
\bibliography{../../org/notes/papers/references}
\end{document}
